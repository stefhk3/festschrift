This work has its origin in two strands of research. One is the simulation of chemical reactions and biological processes. The other is research on process calculi to model computation. These have been merged  to better understand both areas of research. 

In chemistry, speed of reactions and reaction rates have been modelled, with ordinary differential equations (ODEs) being an efficient tool \cite{higham}. Whilst such methods accurately show the dynamic behaviour, it was soon noted that they do not deal with the objects involved. In particular, any properties of the objects, which make reactions possible, are not considered. 
It became even more important when biochemical processes, which do not only involve small molecules, but also macromolecules, cells, and membranes, were modelled. This lack of understanding of the objects was raised in \cite{fontana} and the use of computer science methods like the $\lambda$-calculus was suggested. 

A connection between process calculi and chemistry has been made in \cite{chamjournal}. Here, the Chemical Abstract Machine (CHAM) is introduced as a method to define the semantics of a calculus. As opposed to other semantics for calculi, the algebraic terms (``molecules'') can interact with each other no matter what order they are written in - this is similar to molecules being in a solution. CHAM also introduces membranes to group atoms and rules which simulate heating and cooling of solutions. The actual rules determining the reactions must be defined for a particular CHAM. Therefore, a CHAM can be used to execute processes of arbitrary process calculi. A CHAM can also model actual chemistry depending on the rules used. Starting with Regev et~al.~\cite{regev2000}, process calculi, specifically the $\pi$-calculus, were used to model biochemical systems. There is an analogy between concurrent processes and biological entities in natural systems, and between communication in computer systems and interactions between entities in biological systems. The modelling of biochemical systems is done by representing biological units as processes, and events of binding and unbinding as establishing and breaking a communication, respectively, between processes. The stochastic $\pi$ calculus \cite{PriamiStochasticPi} was used to extend the model to include reaction rates \cite{PriameRegev}. This enables the simulation of the concentrations and of the overall state of the system over time. 

Other work in the area include Bio-PEPA \cite{CiocchettaBiopepa}, which combines a process algebra with timing information and can represent reaction rates. In a Bio-PEPA simulation the entities modelled are species. Every interaction of molecules creates a new species, and there is no tracking of different ``original'' components. The transitions between the entities model the transformations between species. There is no tracking of mechanisms or modelling of the underlying causes. The focus is on the rates and the development of concentrations over time.  

P Systems \cite{psystems} was the first attempt to model membranes and compartments, with objects being able to move in and out of them. A P System contains several membranes, which can be inside one another. For every membrane a set of rules is defined, making P Systems a rule-based system. Objects are located inside membranes, can enter and leave membranes, and membranes can be dissolved. In \cite{CIOBANU2003123} the usage of P Systems for modelling of distributed systems was demonstrated. \cite{CIOBANU2007117} shows that even P systems with minimal parallelism can be universal. An example of a biological application is \cite{10.1007/978-3-540-31837-8_12}, modelling the sodium-potassium pump. 

BioAmbients \cite{RegevBioambients} is based on the Ambient calculus \cite{CARDELLI2000177}, which was originally devised for mobile computing outside the biological context and is an extension of the $\pi$-calculus. Here, similarly to P Systems, processes can be inside compartments.% (and compartments can be in other compartments as well). 
Brane Calculi (``brane'' is short for membrane) \cite{CardelliMobileAmbients}, a family of process calculi, model membranes not only as compartments that define which interactions can happen, but as entities that can be transformed to gain new capabilities. Typical examples are viruses entering cells by interacting with their membranes. The Projective Brane Calculus \cite{ProjectiveBrane} is a variant where interactions are directed outward or inward from a membrane. Other formalims include the Language for Biochemical Systems (LBS) \cite{PlotkinLBS}, the Calculus of Chemical Systems \cite{PlotkinCCS}, and the Biochemical Abstract Machine (BIOCHAM) \cite{biocham}. 

Biochemical reactions are in many cases reversible under certain conditions. The formalisms mentioned do not explicitly model reversibility as undoing of previously performed actions. Instead they use forward actions, which represent undoing of other actions.
%They include potentially actions which are the reverse of other actions, but those reverse actions are independent actions. Logically, they belong together, but this is not modelled. 
The first attempt at the modelling of undoing of forward computation was RCCS~\cite{10.1007/978-3-540-28644-8_19}, a reversible calculus based on CCS. Reversibility is achieved by adding memories to the processes, where a record of past computation is stored. Another reversible calculus based on CCS is CCSK, introduced in \cite{PHILLIPS200770}. It uses keys as a particular form of memory. An extension of CCSK is the Calculus of Covelent Bonding (CCB), which was introduced in~\cite{KU16} and fully defined in~\cite{KU2017}. This enables \textit{locally controlled reversibility} by linking forming and breaking of bonds in the syntax of the calculus. CCB has been applied to chemical and biological modelling \cite{10.1007/978-3-319-99498-7_8, Kuhn2020ReversibilityIC}. Reversibility for the $\pi$-calculus was addressed in \cite{10.1007/978-3-642-15375-4_33}. The $\kappa$-calculus \cite{DANOS200469} is based on graph-rewriting. Here, entities (e.g. proteins) are defined as having several sites, which can potentially bind to other entities. The rewrite rules tell that a link can be formed or broken if the sites in the involved entities are in certain states. Reversibility in rule-based systems is discussed in \cite{Aman2020}.
